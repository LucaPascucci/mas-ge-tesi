\documentclass[12pt,a4paper,italian]{book}


% numera i subsubsection 1 - 1.1 - 1.1.1 - 1.1.1.1
\setcounter{secnumdepth}{3}

%%%%%%%%%%%%%%%%%%%%%%%%%%%%%%%%%%%%%%
%    Scelta dei package da usare     %
%%%%%%%%%%%%%%%%%%%%%%%%%%%%%%%%%%%%%%

\usepackage[italian]{babel}
\usepackage[utf8]{inputenc}
\usepackage[T1]{fontenc}
\usepackage{amsmath,amsfonts,amssymb,amsthm}
\usepackage{deistesi}
\usepackage{fancyhdr}
\usepackage{multirow}
\usepackage{tabularx}
\usepackage{float}

\usepackage[
   pdftex,
   pdfauthor={Luca Pascucci},
   pdftitle={Game Engine come layer di modellazione dell'ambiente per un Sistema Multi-Agente},
   pdfsubject={Relazione Sistemi Autonomi},
   bookmarks=true,
   bookmarksopen=true,
   breaklinks=true,
   colorlinks=true,
   linkcolor=black,
   anchorcolor=blue,
   citecolor=blue,
   filecolor=red,
   urlcolor=blue,
   pdfborderstyle={/S/U/W 1}% border style will be underline of width 1pt
]{hyperref}

\urlstyle{same}

\usepackage[final]{listings}
\usepackage[pdftex,dvipsnames]{xcolor}
\usepackage{inconsolata}

\definecolor{mygray}{gray}{0.6}
\definecolor{delim}{RGB}{20,105,176}
\colorlet{myred}{red!60!black}
\colorlet{numb}{red!60!black}

\lstset{
   frame=bt,
   captionpos=b,
   aboveskip=3mm,
   belowskip=3mm,
   showspaces=false,                % show spaces everywhere adding particular underscores; it overrides 'showstringspaces'
   showstringspaces=false,          % underline spaces within strings only
   showtabs=false,
   columns=flexible,
   extendedchars=true,
   basicstyle=\fontsize{10}{10}\ttfamily,
   numbers=left,
   stepnumber=3,
   numberstyle=\tiny\color{gray},
   keywordstyle=\color{blue},
   commentstyle=\color{ForestGreen},
   stringstyle=\color{mygray},
   breaklines=true,
   breakatwhitespace=true,
   tabsize=2,
   title=\lstname,
   literate=
  {á}{{\'a}}1 {é}{{\'e}}1 {í}{{\'i}}1 {ó}{{\'o}}1 {ú}{{\'u}}1
  {Á}{{\'A}}1 {É}{{\'E}}1 {Í}{{\'I}}1 {Ó}{{\'O}}1 {Ú}{{\'U}}1
  {à}{{\`a}}1 {è}{{\`e}}1 {ì}{{\`i}}1 {ò}{{\`o}}1 {ù}{{\`u}}1
  {À}{{\`A}}1 {È}{{\'E}}1 {Ì}{{\`I}}1 {Ò}{{\`O}}1 {Ù}{{\`U}}1
  {ä}{{\"a}}1 {ë}{{\"e}}1 {ï}{{\"i}}1 {ö}{{\"o}}1 {ü}{{\"u}}1
  {Ä}{{\"A}}1 {Ë}{{\"E}}1 {Ï}{{\"I}}1 {Ö}{{\"O}}1 {Ü}{{\"U}}1
  {â}{{\^a}}1 {ê}{{\^e}}1 {î}{{\^i}}1 {ô}{{\^o}}1 {û}{{\^u}}1
  {Â}{{\^A}}1 {Ê}{{\^E}}1 {Î}{{\^I}}1 {Ô}{{\^O}}1 {Û}{{\^U}}1
  {Ã}{{\~A}}1 {ã}{{\~a}}1 {Õ}{{\~O}}1 {õ}{{\~o}}1
  {œ}{{\oe}}1 {Œ}{{\OE}}1 {æ}{{\ae}}1 {Æ}{{\AE}}1 {ß}{{\ss}}1
  {ű}{{\H{u}}}1 {Ű}{{\H{U}}}1 {ő}{{\H{o}}}1 {Ő}{{\H{O}}}1
  {ç}{{\c c}}1 {Ç}{{\c C}}1 {ø}{{\o}}1 {å}{{\r a}}1 {Å}{{\r A}}1
  {€}{{\euro}}1 {£}{{\pounds}}1 {«}{{\guillemotleft}}1
  {»}{{\guillemotright}}1 {ñ}{{\~n}}1 {Ñ}{{\~N}}1 {¿}{{?`}}1
}

\lstset{
  language=Java,
  morekeywords={final}
  moredelim=[il][\textcolor{mygray}]{@}{\ },
  moredelim=[is][\textcolor{mygray}]{\%\%}{\%\%}
}

\lstdefinelanguage{json}{
    comment=[l]{//},
    literate=
      {0}{{{\color{myred}0}}}{1}
      {1}{{{\color{myred}1}}}{1}
      {2}{{{\color{myred}2}}}{1}
      {3}{{{\color{myred}3}}}{1}
      {4}{{{\color{myred}4}}}{1}
      {5}{{{\color{myred}5}}}{1}
      {6}{{{\color{myred}6}}}{1}
      {7}{{{\color{myred}7}}}{1}
      {8}{{{\color{myred}8}}}{1}
      {9}{{{\color{myred}9}}}{1}
      {:}{{{\color{myred}{:}}}}{1}
      {,}{{{\color{myred}{,}}}}{1}
      {\{}{{{\color{delim}{\{}}}}{1}
      {\}}{{{\color{delim}{\}}}}}{1}
      {[}{{{\color{delim}{[}}}}{1}
      {]}{{{\color{delim}{]}}}}{1},
}

\lstdefinelanguage{asl}{
    comment=[l]{//},
    morestring=**[d][\color{myred}]{"},
    keywords={include}
}

\usepackage{xargs}
\usepackage[colorinlistoftodos,prependcaption]{todonotes}
\newcommandx{\unsure}[2][1=]{\todo[linecolor=red,backgroundcolor=red!25,bordercolor=red,#1]{#2}}
\newcommandx{\change}[2][1=]{\todo[linecolor=blue,backgroundcolor=blue!25,bordercolor=blue,#1]{#2}}
\newcommandx{\improvement}[2][1=]{\todo[linecolor=ForestGreen,backgroundcolor=ForestGreen!25,bordercolor=ForestGreen,#1]{#2}}
\newcommandx{\marianiSays}[2][1=]{\todo[linecolor=Orange,backgroundcolor=Orange!25,bordercolor=Orange,#1]{#2}}

\makeatletter

\newcommand\ProcessThreeDashes{\llap{\color{cyan}\mdseries-{-}-}}

%%%%%%%%%%%%%%%%%%%%%%%%%%%%%%%%%%%%%%%%
% Scelta delle dimensioni della pagina %
%%%%%%%%%%%%%%%%%%%%%%%%%%%%%%%%%%%%%%%%

\setlength{\textwidth}{13.5cm}
\setlength{\textheight}{19cm}
\setlength{\footskip}{3cm}
\setlength{\headheight}{15pt}
\oddsidemargin=50pt \evensidemargin=20pt

%%%%%%%%%%%%%%%%%%%%%%%%%%%%%%%%%%%%%%
%  Informazioni generali sulla Tesi  %
%    da usare nell'intestazione      %
%%%%%%%%%%%%%%%%%%%%%%%%%%%%%%%%%%%%%%

\titolo{Game Engine come layer di modellazione dell'ambiente per un Sistema Multi-Agente}
\laureando{Luca Pascucci}
\annoaccademico{2018--2019}
\sessione{II}
\facolta{CAMPUS DI CESENA\\SCUOLA DI INGEGNERIA E ARCHITETTURA}
\corsodilaurea{Ingegneria e Scienze Informatiche}
\corso{Sistemi Autonomi}
\relatore{Andrea Omicini}
\correlatorea{Stefano Mariani}
%\correlatoreb[]{Pablo Neruda}
\parolechiave{Sistema Multi-Agente}{Game Engine}{Middleware}{Play framework}{...}

\dedica{\textit{"Dedica...}}
%\dedica{\textit{"Non ti dico di correre...\\ma non fermarti mai!"\\L.P.}}

\author{Luca Pascucci}
\title{Game Engine come layer di modellazione dell'ambiente per un Sistema Multi-Agente}
\date{\today}

%%%%%%%%%%%%%%%%%%%%%%%%%%%%%%%%%%%%%
% Fine Preambolo %
% Inizio tesi %
%%%%%%%%%%%%%%%%%%%%%%%%%%%%%%%%%%%%%%

\begin{document}

%%%%%%%%%%%%%%%%%%%%%%%%%
% inizio prefazione
% pagina del titolo, indice, sommario
%%%%%%%%%%%%%%%%%%%%%%%%%

\frontmatter \maketitle \pagestyle{plain} \tableofcontents

\chapter{Sommario}

Con lo sviluppo della tecnologia per creare ambienti virtuali più realistici, complessi e dinamici sta aumentando l'interesse sulle Game Engine (GE), le quali si rivelano sempre più importanti in numerosi ambiti, permettendo lo sviluppo di applicazioni moderne e videogiochi con relativa facilità.

\medskip

Nell'ambito di ricerca, in particolare nel contesto dei Sistemi Multi-Agente (MAS), sono state recentemente utilizzate come supporto alla definizione dell'ambiente e come mezzo abilitante per la coordinazione.\cite{gamemas-woa2016}

\medskip

In questa attività propedeutica alla tesi si analizzerà lo stato dell’arte attuale delle integrazioni tra MAS e GE, ricercando nuove tecnologie utilizzabili con lo scopo finale di realizzare una prima architettura di integrazione, sviluppata successivamente in fase di tesi, utilizzabile per diversi scenari di associazione e comunicazione tra il mondo delle GE ed il mondo dei MAS, lasciando entrambi separati senza modificare le loro astrazioni e funzionalità.


%%%%%%%%%%%%%%%%%%%%%%%%%
% inizio corpo del documento
%
% sequenze dei vari capitoli
% è consigliato mantenere una struttura logica ben definita per separare i vari capitoli
% si consiglia di reificare tale struttura fisicamente sul file system
%%%%%%%%%%%%%%%%%%%%%%%%%

\mainmatter

% stile della pagina
\pagestyle{fancy} \fancyhead[LE,RO]{\bfseries\thepage}

% inclusione dei capitoli
\chapter{Nuova integrazione}

\input{nuova_integrazione.tex}

\section{Stack Tecnologico}

Di seguito vengono illustrate le principali tecnologie prese in esame per la realizzazione di un middleware di collegamento dei due sistemi precedentemente definiti, stabilendo anche quale Sistema Multi-Agente (MAS), Game Engine (GE), framework e tecnologia di collegamento sono stati presi come principale riferimento.

\chapter{JaCaMo} \label{appendice_JaCamo}

\section{Utilizzare la libreria}

La distribuzione della libreria realizzata per JaCaMo viene effettuata attraverso la generazione di un \textit{Jar}\footnote{\textbf{J}ava \textbf{AR}chive} contenente classi Java e file \textit{"asl"}. Questa modalità non permette di includere le dipendenze esterne \cite{gson}\cite{tyrus}, che devono essere importate manualmente nel progetto. 

\medskip

La libreria prodotta, \textit{SynapsisJaCaMo.jar}, è disponibile sul repository del middleware (sezione \ref{materiale_online}) e per utilizzarla è sufficiente importarla nel nuovo progetto JaCaMo.

\section{Realizzare Artefatti Synapsis custom} \label{artefatto_custom}

Come spiegato nella sezione \ref{artefatto_synapsis}, la classe \textit{SynapsisMind} definisce un artefatto con le funzionalità di comunicazione con il middleware. Per realizzare un artefatto custom è sufficiente creare una classe Java che estenda \textit{SynapsisMind}.

\lstinputlisting[label={artefattoSynapsisCustom},caption={Artefatto Synapsis custom},language=Java]{code/TestSynapsisArtifact.java}

Il listato \ref{artefattoSynapsisCustom} contiente un esempio di artefatto custom che estende la classe \textit{SynapsisMind}. Il metodo \textit{init(..)} è il costruttore invocato, secondo le logiche di CArtAgO, durante la creazione di un generico artefatto. Su Synapsis, questo metodo viene utilizzato per fornire all'artefatto informazioni utili (indirizzo e tentativi di riconnessione) per effettuare il collegamento WebSocket, quindi nell'artefatto custom è necessario richiamare il metodo originario utilizzando il costrutto \textit{super}.

\medskip

Con l'estensione della classe \textit{SynapsisMind} viene automaticamente richiesta la definizione dei metodi che permettono la gestione di informazioni in arrivo alla mente:
\begin{itemize}
    \item counterpartEntityReady --> Invocato al collegamento con l'entità corpo;
    \item counterpartEntityUnready --> Invocato alla disconnessione con l'entità corpo;
    \item parseIncomingMessage --> Invocato ad ogni messaggio (percezione) ricevuto dal corpo;
\end{itemize}

Per definire azioni specifiche, in aggiunta alle azioni già presenti spiegate nella sezione \ref{artefatto_synapsis}, è necessario definire un nuovo metodo, con la notazione @OPERATION (per renderlo utilizzabile dall'agente), ed invocare il metodo \textit{doAction} per inviare il messaggio al corpo. Nel listato  \ref{artefattoSynapsisCustom} è presente il metodo \textit{azionePersonalizzata} come esempio di azione custom.

\medskip

Per la modalità di collegamento tra artefatto ed agente è necessario utilizzare anche la successiva sezione.

\section{Realizzare Agenti Synapsis custom} \label{agente_custom}

Per realizzare agenti Synapsis custom è necessario importare l'agente \textit{synapsis\_base\_agent.asl} presente nella libreria. 

\lstinputlisting[label={jarasl},caption={Includere file .asl da Jar},language=asl]{code/includereASLdaJAR.asl}

Nel listato \ref{jarasl} è presente il comando, messo a disposizione da JaCaMo, per importare file .asl presenti all'interno di Jar. Questa riga è da aggiungere al nuovo agente che si sta realizzando. Con l'operazione precedente si hanno a diposizione tutti i beliefs, goals e plan di \textit{synapsis\_base\_agent}.

\lstinputlisting[label={testagent},caption={Agente Synapsis custom},language=asl]{code/test_agent.asl}

Il listato \ref{testagent} rappresenta un agente custom pronto all'utilizzo. I beliefs iniziali sono necessari ad effettuare il collegamento WebSocket. Da notare è il belief \textit{synapsis\_mind\_class("artifacts.TestSynapsisArtifact")} utilizzato da \textit{synapsis\_base\_agent} per sapere quale classe rappresenta il proprio artefatto.

\medskip

Il plan \textit{+synapsis\_counterpart\_status(Name, C)} viene utilizzato per conoscere lo stato di collegamento con la controparte corpo. Questo plan viene scatenato dall'aggiornamento delle proprietà osservabili dell'artefatto secondo le informazioni ricevute dal middelware. Per completare il ciclo è stato definito il plan \textit{+!prova} per utilizzare l'operazione precedentemene definita nell'artefatto \ref{artefattoSynapsisCustom}.

\subsection{Play} \label{play}
Play è un framework lightweight, stateless e asincrono per la creazione di applicazioni e servizi Web. È stato costruito utilizzando Scala e Akka e mira a fornire gli strumenti per la realizzazione di applicazioni altamente scalabili con consumo minimo di risorse, ad esempio CPU, memoria, thread \cite{play-framework}.

\medskip

Play incorpora un HTTP Server integrato (quindi non è necessario un server di separato come in molti Web Framework Java), un modello per la realizzazione di applicazioni baste su servizi RESTful e mette a disposizioni strumenti per la gestione di Form, protezione CSRF\footnote{Cross-Site Request Forgery} e meccanismi di instradamento. Per semplificare il suo utilizzo fa largo uso del pattern Model-View-Controller, comune e facilmente utilizzabile, fornendo paradigmi di programmazione concisi e funzionali.

\begin{figure}[H]
\centering
\includegraphics[width=\textwidth]{figures/Play_structure.png}
\caption{Struttura MVC di un'applicazione realizzata con Play \cite{play_framework_book}}
\end{figure}

Lo stack nelle Web Application nel mondo Java Enterprise è basato su una tecnologia che si è evoluta nel corso degli anni e richiede diversi elementi (strati) per funzionare. E' molto probabile che le molteplici tecnologie che comprendono questo stack rendano anche l'implementazione di semplici applicazioni problematica e soggetta a errori poiché ogni tecnologia deve essere integrata con successo con la successiva, spesso basandosi su file di configurazione o convenzioni standard \cite{play_framework_book}.

\begin{figure}[H]
\centering
\includegraphics[width=5cm]{figures/Java_EE_layered_architecture.png}
\caption{Architettura a strati JavaEE \cite{play_framework_book}}
\label{Java_EE_layered_architecture}
\end{figure}

Il framework Play è stato progettato per diminuire lo stack (Figura \ref{Java_EE_layered_architecture})  richiedendo l'utilizzo di un solo server HTTP per funzionare.

\begin{figure}[H]
\centering
\includegraphics[width=5cm]{figures/Play_layered_architecture.png}
\caption{Architettura a strati Play framework \cite{play_framework_book}}
\label{Play_layered_architecture}
\end{figure}

Lo strato Play (Figura \ref{Play_layered_architecture}) è formato da una serie di componenti che includono:
\begin{itemize}
    \item HTTP Server: componente che riceve la richiesta HTTP da un client e restituisce un risultato basato sulle informazioni fornite nella richiesta;
    \item Router: determina dove instradare la richiesta, pertanto fornisce un file di configurazione dei percorsi disponibili nell'applicativo;
    \item Sistema di templating HTML dinamico: Utilizza pagine standard in HTML e le popola con dati generati dinamicamente dall'applicazione;
    \item Console integrata: Per semplificare l'utilizzo di Play, viene fornita una suite di strumenti che possono essere utilizzati per creare, aggiornare e distribuire l'applicazione Play. Questi strumenti sono accessibili e gestiti dalla console;
    \item Persistent framework: Funzionalità utili per accesso a database.
\end{itemize}
\subsubsection{Akka - Modello ad attori}

Akka è un toolkit per la creazione di applicazioni altamente distribuite, concorrenti, event-driven, tolleranti ai guasti. Play framework utilizza il modello ad attori presente in Akka, dove l'attore è l'entità principale, per aumentare il livello di astrazione e fornire una piattaforma per la realizzazioni di applicazioni concorrenti e scalabili \cite{akka}.

\medskip

Il modello ad attori (che risale al 1973) si basa sull'idea di avere attori simultanei indipendenti che ricevono e inviano messaggi asincroni e che svolgono un comportamento basato su questi messaggi. Gli attori possono mantenere il proprio stato e comportamento. Tuttavia, idealmente solo i dati immutabili vengono scambiati tra di loro, pertanto ogni attore è indipendente da tutti gli altri ed esegue solo alcuni calcoli o elaborazioni basati su un messaggio ricevuto da esso.

\medskip

L'idea chiave alla base del modello ad attori è che la maggior parte dei problemi come concorrenza, deadlock, corruzione dei dati, derivino dalla condivisione dello stato. Pertanto, nel mondo degli attori non esiste uno stato condiviso (come una coda concorrente produttore-consumatore). Al contrario, i messaggi vengono inviati tra attori e questi messaggi vengono messi in coda in una casella di posta in modo simile ai messaggi di posta elettronica \cite{play_framework_book}.

\begin{figure}[H]
\centering
\includegraphics[width=7cm]{figures/Actors_communicating.png}
\caption{Comunicazione tra attori attraverso messaggi asincroni \cite{akka}}
\end{figure}


\subsection{WebSocket} \label{standard_websocket}

Il protocollo WebSocket (WS) consente la comunicazione bidirezionale tra un client a un host remoto instaurando un canale di comunicazione utilizzabile da entrambi sia in scrittura che in lettura. Il modello di sicurezza utilizzato è l'origin-based security model comunemente usato dai browser web. Il protocollo consiste in una fase di hand-shake di apertura seguita dal successivo invio/ricezione di una serie di messaggi strutturati, stratificata su una connessione TCP\footnote{Transmission Control Protocol} persistente. L'obiettivo di questa tecnologia è fornire un meccanismo per le applicazioni basate su browser che necessitano di comunicazione bidirezionale in tempo reale con server che non si basano sull'apertura di più connessioni HTTP \cite{RFC6455}.

\subsubsection{Introduzione delle WebSocket}

Storicamente, la creazione di applicazioni Web che richiedono la comunicazione bidirezionale tra client e server (ad esempio applicazioni di messaggistica istantanea e giochi) ha portato ad un abuso di HTTP utilizzato per operazioni di polling (verifica ciclica) verso il server con lo scopo di controllare gli aggiornamenti, inviando notifiche upstream come chiamate HTTP distinte \cite{RFC6202}.

\medskip

Il protocollo HTTP è stato inteso fin dalla prima versione ideata da Tim Berners-Lee come metodo per recuperare risorse remote in maniera semplice: una richiesta per ogni pagina Web, ogni immagine o l’invio di dati da rendere persistente. Con il passare degli anni però, all’incirca intorno al 2004, lo sviluppo di applicazioni Web subì una forte accelerazione dovuta all’introduzione di una nuova tecnologia, Ajax, che grazie all'utilizzo di Javascript fu in grado di creare e gestire richieste HTTP asincrone tramite funzioni di callback dedicate.

\medskip

Seguendo l’evoluzione delle applicazioni Web molte applicazioni prevedevano una user experience orientata al real-time. Esempi possono essere applicazioni di chat, videogiochi multiplayer o sistemi di notifiche, tutte applicazioni che la sola tecnologia Ajax (o simili, come connessioni HTTP persistenti COMET) poteva simulare solo in parte con sistemi di polling poco performanti e complessi da implementare.

\medskip

La soluzione arrivò quando ci si rese conto che la risposta a questi problemi risiedeva effettivamente nel protocollo stesso: HTTP sfrutta a livello di rete il protocollo TCP/IP, connection-oriented, usata in altri contesti singolarmente per connessioni full-duplex. Nacque così il protocollo WebSocket con un ottimo tempismo considerando l’avvento contemporaneo di HTML5 e altre tecnologie che contribuiranno successivamente alla diffusione del protocollo \cite{websocket_hystory}.

\medskip

\subsubsection{Principali caratteristiche}

Ecco le caratteristiche principali del protocollo WebSocket:
\begin{itemize}
    \item \textbf{Bidirezionali}: quando il canale di comunicazione è attivo, sia il client che il server sono connessi ed entrambi possono inviare e ricevere messaggi;
    \item \textbf{Full-duplex}: i dati inviati contemporaneamente dai due attori (client e server) non generano collisioni e vengono ricevuti correttamente;
    \item \textbf{Basati su TCP}: il protocollo usato a livello di rete per la comunicazione è il TCP, che garantisce un meccanismo affidabile (controllo degli errori, re-invio di pacchetti persi, ecc) per il trasporto di byte da una sorgente a una destinazione;
    \item \textbf{Client-key handshake}: All’apertura di una connessione, il client invia al server una chiave segreta di 16 byte codificata con base64. Il server aggiunge a questa un’altra stringa, detta GUID\footnote{Globally Unique Identifier} specificata nel protocollo e codifica con SHA1 e invia il risultato al client. Cosi facendo, il client può verificare che l’identità del server che ha risposto corrisponda a quella desiderata;
    \item \textbf{Sicurezza origin-based}: Alla richiesta di una nuova connessione, il server può identificare l’origine della richiesta come non autorizzata o non attendibile e rifiutarla.
    \item \textbf{Maschera dei dati}: Nella trama iniziale di ogni messaggio, il client invia una maschera di 4 byte per l’offuscamento. Effettuando uno XOR bit a bit tra i dati trasmessi e la chiave è possibile ottenere il messaggio originale. Ciò è utile per evitare lo sniffing, cioè l’intercettazione di informazioni da parte di terze parti.
\end{itemize}

\subsubsection{WebSocket URIs}

La specifica RFC \cite{RFC6455} stabilisce due differenti tipologie di URI\footnote{Uniform Resource Identifier} per rappresentare una risorsa remota di tipo WebSocket:

\begin{enumerate}
    \item ws-URI = ws://HOST[:PORT]/PATH[?QUERY];
    \item wss-URI = wss://HOST[:PORT]/PATH[?QUERY].
\end{enumerate}

In entrambi i casi i seguenti parametri stanno a significare:

\begin{itemize}
    \item HOST: l'host dove risiede la risorsa;
    \item PORT: la porta sul quale l'host rende disponibile la risorsa. Questo pramentro è opzionale e se non specificato viene inteso:
    \begin{itemize}
        \item Porta 80 per connessioni WS
        \item Porta 443 per connessioni WSS
    \end{itemize}
    \item PATH: il percorso all'interno dell'host dove trovare la risorsa;
    \item QUERY: la query da compiere sulla risorsa.
\end{itemize}


L'utilizzo di questo protocollo per lo svolgimento della tesi è stato favorito dalle caratteristiche coerenti al tipo di sistema che si intende creare, nonché dal supporto nativo presente nel framework Play e dalla facile reperibilità di librerie per JaCaMo e Unity:

\begin{itemize}
   \item \textbf{websocket-sharp}: Implementazione C\# del protocollo WebSocket client/server \cite{websocket-sharp}
   \item \textbf{project tyrus}: Implementazione Java dello standard JSR 356\footnote{Java API per WebSocket, conforme al protocollo RFC6455} \cite{tyrus}
\end{itemize}

\subsection{Unity} \label{unity}

Unity è una Game Engine (GE) cross-platform sviluppata da Unity Technologies, utilizzata per la creazione di videogiochi (sia 2D che 3D) e simulazioni, che supporta la distribuzione su una larga varietà di piattaforme (PC, console, dispositivi mobili, etc.). Fornisce astrazioni che contribuiscono ad estendere il suo utilizzo tra gli sviluppatori e programmatori, rendendola una delle GE più utilizzate per produrre in maniera veloce ed efficace applicazioni e giochi \cite{unity}.

\medskip

Inoltre, questa GE supporta molte funzionalità facili da utilizzare e sfruttabili per creare giochi realistici e simulazioni immersive, come un intuitivo editor real-time, un sistema di fisica integrato, luci dinamiche, la possibilità di creare oggetti 2D e 3D direttamente dall'IDE o di importarli esternamente, gli shader, un supporto per l'intelligenza artificiale (capacità di evitare gli ostacoli, ricerca del percorso, etc.), e cosi via.

\medskip

Le funzionalità principali messe a disposizione del designer sono:
\begin{itemize}
	\item \textbf{GameObject}: La classe base per tutte le entità presenti su una scena di Unity: un personaggio controllabile dall'utente, un personaggio non giocabile, un oggetto (2D/3D). Tutto ciò che è presente sulla scena è un GameObject.
	\item \textbf{Script}: Codice sorgente applicato a un GameObject, grazie al quale è possibile assegnare a quest'ultimo comportamenti e proprietà dinamiche. Gli script vengono eseguiti dal game loop di Unity, che in maniera sequenziale esegue una volta ogni script, durante ogni frame del gioco. Non esiste concorrenza. Il comportamento è il risultato della logica definita nello script attraverso funzioni e routine. Le proprietà equivalgono a variabili che possono essere manipolate nello script oppure definite dall'IDE grafico.
	\item \textbf{Component}: Elemento, proprietà speciale assegnabile ai GameObject. A seconda del tipo di GameObject che si desidera creare è necessario aggiungere diverse combinazioni di Components. I Components basilari riguardano la fisica (Transform, Collider,..), l'illuminazione (Light) e la renderizzazione del GameObject (Render). \'E possibile istanziare runtime Components attraverso gli script.
	\item \textbf{Coroutine}: Una soluzione alla sequenzialità imposta agli script, grazie al quale è possibile partizionare una computazione e distribuirla su più frame, sospendendo e riprendendo l'esecuzione in precisi punti del codice.
	\item \textbf{Prefab}: Rappresentazione di un GameObject complesso, completo di Script e Component, istanziabile più volte a run-time. Le modifiche della struttura, proprietà e componenti del Prefab si propagheranno a tutti i GameObject collegati allo stesso presenti nella scena di gioco.
	\item \textbf{Event e Messaging System}: sistema ad eventi utile per far comunicare tra loro diversi GameObject. Questi sistemi sono formati tipicamente da eventi e listener. I listener si sottoscrivono ad eventi di un certo tipo; quando l'evento si verifica, viene notificato a tutti i listener in ascolto dello stesso tipo attraverso l'invio di un messaggio.
\end{itemize}

\section{Realizzare un oggetto in Unity} \label{ambiente_unity}

Gli strumenti a disposizione permettono agli sviluppatori di realizzare qualunque tipo di oggetto: dai più semplici, come un cubo, ai più articolati, ad esempio un robot. 

\medskip

Per realizzare oggetti complessi è possibile definire una gerarchia di componenti e dotare ognuno di loro delle stesse funzionalità dell'oggetto padre. Ripensando all'esempio del robot, lo sviluppatore può suddividere lo stesso in sotto-componenti più articolate, quali testa, braccia, addome, gambe, fino ad arrivare a realizzare parti basilari quali dita, occhi e così via.

\begin{figure}[H]
\centering
\includegraphics[width=\textwidth]{figures/unity_diagram.png}
\caption{GameObject e Components}
\end{figure}

\subsection{Muovere il GameObject}

Come spiegato precedentemente, ogni oggetto presente su Unity è un GameObject e, per definizione, contiene le seguenti proprietà fondamentali: 
\begin{itemize}
    \item Posizione
    \item Scala
    \item Rotazione
\end{itemize}
Le proprietà appena elencate sono elementi fondamentali del "component" Transform \cite{unity_transform}, automaticamente realizzato per ogni oggetto in scena.

\medskip

Attraverso l'associazione di uno script al GameObject, lo sviluppatore può accedere ad ogni proprietà di quest'ultimo. In tal modo è possibile modificare runtime la sua posizione, la scala e la rotazione e, applicando diverse tipologie di trasformazioni, lo si anima. Questa modalità di animazione è basilare, ma sufficiente per l'obiettivo finale posto. Sono poi presenti meccanismi complessi in caso di elaborazioni più articolate e specifiche come, ad esempio, la simulazione della corsa umana.

\subsection{Fisicità del GameObject}

La semplicità nella realizzazione di oggetti all'interno delle Game Engine è affiancata alla presenza di un motore fisico: attraverso quest'ultimo, Unity elabora e modifica dinamicamente ogni oggetto in scena in base alle specifiche fisiche ad esso attribuite. Il motore fisico rende possibile la simulazione di forza di gravità, la fisica dei movimenti e le occlusioni della scena.

\medskip

In Unity, per definire la fisicità di un GameObject, è necessario attribuirgli uno specifico "component" chiamato RigidBody \cite{unity_rigidbody}. Aggiungendo questo componente, il movimento del GameObject nella scena è controllato dal motore fisico di Unity. Di questo componente è possibile specificare:
\begin{itemize}
    \item Massa
    \item Resistenza
    \item Velocità di movimento
    \item Soggezione alla gravità
\end{itemize}

\subsection{Percepire l'ambiente}

La percezione generalmente è associata all'acquisizione di una realtà interna o esterna attraverso l'elaborazione organica e psichica di stimoli sensoriali \cite{treccani}. Nelle Game Engine, rendere un oggetto capace di percepire la realtà è spesso collegato a renderlo fisicamente consapevole della propria superficie.

\medskip

Su Unity è presente il "Collision Detection System", il quale controlla ogni evento di interazione fisica tra due o più GameObject nella scena e, attraverso l'aggiunta del "component" Collider al GameObject, viene specificato che l'oggetto deve essere preso in considerazione dal sistema durante la generazione di eventi di collisione.

\medskip

Il Collider è associabile ad un GameObject, più o meno complesso, ed è capace di creare un'area generica, come ad esempio un cubo/sfera, che circonda l'oggetto, oppure mappare alla perfezione la sua superficie. Gli eventi di interazione creati dal "Collision Detection System" sono utilizzabili, da parte dello sviluppatore, nello script collegato al GameObject, difatti durante una collisione vengono invocati degli specifici metodi all'interno dello script con tutte le informazioni sull'evento emesso \cite{unity_collision}.

\medskip

L'ultimo passaggio è fondamentale, dato che permette di completare il processo di percezione dell'ambiente da parte di un generico GameObject e, quindi, lo rende consapevole della propria presenza nella scena.

\medskip

Per aumentare la capacità di percezione del GameObject, all'interno di Unity ogni oggetto può essere visto e utilizzato da ogni altro oggetto in scena. In questa maniera oltre alla percezione del proprio corpo fisico, il GameObject è in grado di conoscere anche quali altri elementi compongono la scena, ottenendo quindi la percezione totale dell'ambiente.

\medskip

Tutte le procedure sopra illustrate sono replicabili per ogni GameObject presente in scena. In questo modo è possibile realizzare scene più o meno complesse.

\subsection{Modificare l'ambiente}

Il GameObject, come illustrato in precedenza, è la classe base di tutti gli oggetti presenti su una scena Unity, di conseguenza, l'ambiente stesso è un GameObject (più o meno complesso). Questo concetto, unito alla possibilità di ogni GameObject di interagire sia fisicamente che logicamente (script) con ogni altro oggetto in scena, dà luogo ad infinite possibilità di modifica della scena. Ad esempio, in caso di collisione tra due oggetti, il motore fisico, unito al motore grafico, calcola il possibile spostamento degli stessi che quindi porta ad un'effettiva modifica dell'ambiente.


\chapter{Background}

\section{Stato dell'arte}

Come premessa al lavoro svolto, si presenta una prima introduzione ai Sistemi Multi-Agente (MAS), alle Game Engine (GE) ed alle integrazioni già realizzate, spiegando brevemente le astrazioni presenti nei MAS. Si vedrà un breve excursus storico sulle GE e le diverse tipologie di soluzioni già realizzate.

\subsection{Sistema Multi-Agente} \label{sistema_multi-agente}

La crescente complessità nell'ingegnerizzazione dei sistemi software ha portato alla necessità di modelli e astrazioni in grado di rendere più facile la loro progettazione, lo sviluppo e il mantenimento. In questa direzione, la computazione orientata agli agenti viene in aiuto agli ingegneri ed informatici per costruire sistemi complessi, virtuali o artificiali permettendo una loro agevole e corretta gestione \cite{mas-as-complex-systems}.

\medskip

In particolare, la ricerca e le tecnologie per MAS hanno introdotto nuove astrazioni per affrontare la complessità durante la progettazione di sistemi o applicazioni composte da individui che non agiscono più da soli ma all'interno di una società. Le tecnologie e i modelli agent-oriented sono attualmente diventati una potente tecnica in grado di affrontare molti problemi che vengono alla luce durante la progettazione di sistemi computer-based in termini di entità che condividono caratteristiche quali l'autonomia, l'intelligenza, la distribuzione, l'interazione, la coordinazione, etc.

\medskip

L’ingegnerizzazione dei MAS si occupa infatti di costruire sistemi complessi dove più entità autonome chiamate agenti cercano di raggiungere in maniera proattiva i loro scopi sfruttando le interazioni tra di essi (come una società), e con l'ambiente circostante. Questo modello può essere visto come un paradigma general-purpose, il quale prevede l'utilizzo di tecnologie agent-oriented in diversi scenari applicativi \cite{aose-jaamas9}.

\medskip

Un MAS fornisce agli sviluppatori e ai designer tre astrazioni principali:
\begin{itemize}
    \item Agenti: Le entità autonome che compongono il sistema. Sono in grado di comunicare e possono essere intelligenti, dinamici, e situati;
    \item Società: Rappresenta un gruppo di entità il cui comportamento emerge dall'interazione tra i singoli elementi;
    \item Ambiente: Il "contenitore" in cui gli agenti sono immersi e con il quale questi ultimi possono interagire, modificandolo. La caratteristica degli agenti di essere situati nell'ambiente in cui si trovano permette loro di percepire e produrre cambiamenti su di esso.
\end{itemize}


\section{Game Engine}
Le Game Engine (GE) sono framework utilizzati per supportare la progettazione e lo sviluppo di giochi. Il termine "Game Engine" nacque a metà degli anni '90 in riferimento a giochi sparatutto in prima persona (FPS) come il popolare "Doom" progettato con una separazione ragionevolmente ben definita tra i suoi componenti software principali (come il sistema di rendering grafico tridimensionale, il sistema di rilevamento delle collisioni o il sistema audio) e le risorse artistiche, i mondi di gioco e le regole di gioco che comprendevano l'esperienza di gioco del giocatore.

\medskip

Le GE moderne sono strutture general-purpose multipiattaforma orientate verso ogni aspetto della progettazione e dello sviluppo del gioco, come il rendering 2D/3D delle scene di gioco, i motori fisici per la dinamica ambientale (movimenti, dinamica delle particelle, rilevamento delle collisioni, prevenzione degli ostacoli, ecc.), suoni, script comportamentali, intelligenza artificiale dei personaggi e molto altro.

\medskip

Come esempio significativo che rappresenta la gamma di piattaforme disponibili, nella sezione successiva verrà esaminata una delle più popolari GE - Unity\cite{unity} - con l'obiettivo di:
\begin{itemize}
    \item rilevare quelle astrazioni e quei meccanismi che hanno più probabilità di avere una controparte nel MAS, o almeno quelli che sembrano fornire un supporto nel riformulare le astrazioni mancanti del MAS;
    \item evidenziare le opportunità per colmare le lacune concettuali / tecniche che ostacolano l'integrazione dei due mondi.
\end{itemize}


\subsection{Integrazione}

Sono già presenti esempi di integrazione tra GE e MAS che concentrano la propria attenzione su obiettivi specifici a livello tecnologico, piuttosto che sulla creazione di un'infrastruttura orientata agli agenti basata sul gioco per scopi generici. Per esempio:

\begin{itemize}
    \item QuizMASter \cite{5763564} concentrato sull'astrazione degli agenti collegando gli agenti MAS ai personaggi dei motori di gioco, nel contesto dell'apprendimento educativo
    \item CIGA \cite{ciga} considera sia la modellazione degli agenti che quella dell'ambiente, per agenti virtuali generici in ambienti virtuali
    \item GameBots \cite{gamebots} concentrato sull'astrazione dell'agente, ma considera anche l'ambiente fornendo un framework di sviluppo e un runtime per i test di sistemi multi-agente in ambienti virtuali
    \item UTSAF \cite{utsaf} si concentra sulla modellistica ambientale nel contesto di simulazioni distribuite in ambito militare\footnote{Gli agenti vengono considerati, ma solo come mezzo di integrazione tra diverse piattaforme di simulazione, non nel contesto del GE sfruttato per il rendering di simulazione}
\end{itemize}

Sebbene rappresentino chiaramente esempi di integrazione (parzialmente) riuscita di MAS in GE, i lavori sopra elencati presentano alcune carenze rispetto all'obiettivo che perseguiamo in questo documento.

\smallskip

Solamente CIGA rappresenta un'eccezione che riconosce il divario concettuale tra MAS e GE, e propone soluzioni per affrontarlo (anche se a livello tecnologico). L'unico strato preso in considerazione nel perseguimento dell'integrazione è quello tecnologico - nessun modello, nessuna architettura, nessun linguaggio. All'interno di QuizMASter, UTSAF e GameBot (in una certa misura) l'integrazione è realizzata per specifico obiettivo, e la maggior parte degli approcci fornisce ai programmatori alcune astrazioni per trattare con agenti e ambiente, ma nessuna attenzione viene data alle astrazioni sociali \cite{gamemas-woa2016}.


\section{Situazione iniziale}

I lavori precedentemente svolti, che hanno contribuito alla definizione di questo percorso, utilizzano due approcci nettamente separati per l'integrazione MAS e GE:
\begin{enumerate}
	\item Integrazione delle caratteristiche dei MAS all'interno della GE;
	\item Realizzazione di un middleware, come layer software, per collegare l'ambiente MAS con GE.
\end{enumerate}

\subsection{MAS all'interno di GE} \label{MAS_dentro_GE}

Il primo punto è stato realizzato implementando due modelli tipici dei MAS:
\begin{itemize}
	\item Il modello Beliefs, Desires, Intentions (BDI) per la programmazione degli agenti \cite{amslaurea15657};
	\item Un modello di coordinazione degli agenti tramite spazio di tuple e primitive Linda \cite{amslaurea8424}\cite{amslaurea16100}.
\end{itemize}

Il cuore pulsante di entrambi i lavori risiede nell'uso intensivo di un interprete Prolog fatto ad hoc per Unity, UnityProlog \cite{unity_prolog}. Questo interprete dispone di molte funzionalità per estendere l'interoperabilità di Prolog con i GameObject.
Dal momento che è stato progettato per essere usato in maniera specifica con Unity, nasce con delle primitive che permettono di accedere e manipolare GameObject e i relativi componenti direttamente da Prolog. UnityProlog introduce tuttavia alcune limitazioni da tenere bene in considerazione \cite{amslaurea15657}, anche se allo stato attuale è l'unica versione di Prolog del quale è stato dimostrato il corretto funzionamento:
\begin{itemize}
	\item Un interprete per Prolog non sarà mai performante quanto lo può essere un compilatore e questo può rappresentare un problema per simulazioni di MAS più grandi.
	\item Utilizza lo stack C\# come stack di esecuzione, quindi la tail call optimization non è ancora supportata.
	\item Non supporta regole con più di 10 subgoal, quindi a fronte di una regola complessa con tanti goal da controllare, è necessario frammentare la regola in questione in sotto regole con non più di 10 subgoal per ognuna.
\end{itemize}

\subsection{MAS e GE separati} \label{MAS_GE_separati}

Il secondo percorso si differenzia dal primo per la scelta di lasciare separati GE da MAS realizzando un canale di comunicazione tra i due ambienti. \'E stata introdotta una terminologia per contraddistinguere le entità realizzate sul GE (GameObject) e su MAS (agenti), rispettivamente definite "corpi" e "menti" virtuali. \cite{amslaurea12270}

\medskip

Fondamentalmente un corpo deve eseguire azioni e, a seguito di determinati eventi, deve trasmettere le proprie percezioni alla mente, quest'ultima, invece, deve elaborare le percezioni per decidere quali azioni far svolgere al proprio corpo. Per rendere possibile questa comunicazione è stato progettato e implementato un sistema middleware definito secondo il seguente schema.

\begin{figure}[H]
\centering
\includegraphics[width=9cm]{figures/Middleware_fuschini.png}
\caption{Il middleware viene suddiviso in due parti, poste sui due lati del canale di comunicazione. \cite{amslaurea12270}}
\end{figure}

Dalla figura si può notare la separazione del middleware nei due sistemi, motivato dalle diverse tecnologie utilizzate dai due ambienti.
Questa divisione vincola la realizzazione di una nuova parte di middleware in caso di utilizzo di un diversa tipologia di GE e/o MAS.
\medskip

Il protocollo di comunicazione tra le entità è stato realizzato utilizzando messaggi strutturati. Da una parte, le menti devono definire quale azione deve compiere il relativo corpo (es. "muoviti in avanti", "ruota", "prendi", ecc.), dall'altro i corpi devono far sapere alle relative menti le proprie percezioni dell'ambiente circostante (es. "mi ha toccato un'entità", "sono alle coordinate 23,12,-6", ecc.).\cite{amslaurea12270}

\medskip

Il progetto di tesi è stato realizzato basandosi sul secondo approccio, ma differenziandosi per scelte architetturali e tecnologiche.




\chapter{Synapsis}

\section{Terminologia}

La sinapsi (o giunzione sinaptica) (dal greco synàptein, vale a dire "connettere") è una struttura altamente specializzata che consente la comunicazione delle cellule del tessuto nervoso tra loro (neuroni) o con altre cellule (cellule muscolari, sensoriali). Nello specifico la sinapsi neuromuscolare rappresenta la giunzione tra neurone motore e muscolo a livello della placca motrice, ove ha luogo la trasmissione dell'impulso con le modalità delle sinapsi chimiche: lo spazio extracellulare della sinapsi neuromuscolare è detto chiave sinaptica \cite{treccani}.
La semplice associazione tra l'obiettivo di questo percorso e la parola sopra definita ha portato a denominare il middleware "Synapsis"\footnote{Traduzione in inglese del termine italiano sinapsi.}.

\subsection{Entità}

Successivamente nella trattazione verrà fatto uso del termine "entità" che generalmente viene intesa come insieme di elementi dotati di proprietà comuni dal punto di vista dell’applicazione considerata \cite{treccani}.
Concettualmente, in questo dominio, l'entità viene intesa come oggetto divisibile in due parti, mente e corpo, che collegate riescono a trasmettersi informazioni, utilizzate dalla mente per raggiungere i propri obiettivi e dal corpo per diventare "attivo" nell'ambiente in cui si trova.

\subsection{Mente}

La nozione di mente può essere caratterizzata da alcuni punti chiave fondamentali:
\begin{itemize}
   \item autonomia;
   \item interazione;
   \item obiettivi.
\end{itemize}
In altre parole, una mente può essere pensata come un componente software autonomo che interagisce con l'ambiente per svolgere i propri compiti.
I punti sopra elencati rendono facile l'associazione della mente al concetto di Agente, spiegato nella sezione \ref{sistema_multi-agente}, poiché questa entità del Sistema Multi-Agente (MAS) ingloba astrazioni simili a quelle illustrate nella sezione \ref{jason}.

\subsection{Corpo}

Corpo è un termine generico che indica qualsiasi porzione limitata di materia, cui si attribuiscono, in fisica, le proprietà di estensione, divisibilità, impenetrabilità \cite{treccani}.
In questa trattazione è associabile alla nozione di GameObject di Unity, spiegata nella sezione \ref{unity}, utilizzata per avere una rappresentazione fisica dell'entità da realizzare.

\subsection{Azione}

Nel suo significato più generale è intesa come attività od operazione posta in essere da un determinato soggetto \cite{treccani}.
In questo studio, si considera come "azione" un certo gesto richiesto dalla mente che può essere associato ad una operazione eseguita dal corpo, ad esempio, nel
caso di un'azione del tipo \textit{"vai a (posizione)"}, richiesta dalla mente, corrisponde il movimento del corpo nell'ambiente verso la posizione indicata.

\subsection{Percezione}

La percezione è un atto cognitivo mediato dai sensi con cui si avverte la realtà di un determinato oggetto e che implica un processo di organizzazione e interpretazione \cite{treccani}.

\medskip

In questo lavoro, la percezione si collega ad una certa sensazione rilevata dal corpo ed inviata alla mente per portarla a conoscenza di questa nuova informazione, ad esempio, nel caso del raggiungimento della posizione richiesta in precedenza, il corpo trasmette la percezione \textit{"arrivato (posizione)"} che informa la mente del completamento dell'operazione.

\medskip

Esiste inoltre, da parte del corpo, la possibilità di inviare percezioni "libere" ossia non associate a risposta di un'azione inviata dalla mente. Un semplice esempio è il contatto del corpo con una qualsiasi altra entità nell'ambiente che corrisponde all'invio di una percezione del tipo \textit{"toccato(nome\_entità)"}.

\subsection{Struttura di un'entità} \label{struttura_entita1}

\begin{figure}[H]
   \centering
   \includegraphics[width=8cm]{figures/Entita_struttura.png}
   \caption{Struttura di una generica entità}
   \label{entita_struttura}
\end{figure}

La figura \ref{entita_struttura} rappresenta la struttura di una generica entità, dove:

\begin{itemize}
   \item Il corpo esegue azioni e, in risposta a queste ultime, oppure, a seguito di determinati eventi esterni, trasmette le proprie percezioni alla mente.
   \item La mente elabora le percezioni per decidere quali azioni far svolgere al proprio corpo.
\end{itemize}




\section{Architettura}

\begin{figure}[H]
   \centering
   \includegraphics[width=\linewidth]{figures/Architettura_alto_livello.png}
   \caption{Architettura ad alto livello}
\end{figure}

L'immagine mostra la struttura del sistema realizzato per mettere in comunicazione MAS e GE attraverso l'introduzione di un middleware, realizzato con il framework Play, separato e autonomo rispetto alle differenti tecnologie utilzzate su MAS e GE.

\medskip

Per collegare al middleware i due sistemi, sono state realizzate due librerie, definite nell'immagine sovrastante "Libreria Synapsis", contenenti funzionalità di collegamento e comunicazione con Synapsis.
Le librerie rispettano astrazioni e modelli computazionali di entrambi i sistemi (MAS e GE) ed utilizzano la terminologia precedentemente elencata.

\begin{figure}[H]
   \centering
   \includegraphics[width=\linewidth]{figures/Middleware_entity.png}
   \caption{Divisione di un'entità nel sistema}
\end{figure}

Aggiungendo un'entità alla struttura precedente è possibile notare come la stessa risulti suddivisa tra i due sistemi (MAS e GE) ed, attraverso il collegamento al middleware, venga reso possibile lo scambio di informazioni (percezioni, azioni) pur essendo computazionalmente separate.

\section{Scenario d'esempio}

Per meglio comprendere l'architettura di sistema appena descritta e le interazioni tra i suoi componenti costituenti, si prende a riferimento uno scenario d'esempio chiamato "recycling robots". Come si può intuire dal nome, la scena contiene dei robot, i quali hanno il compito di riciclare la spazzatura presente nell'ambiente portandola in un bidone.

\medskip

Il compito generale di un robot è divisibile in un ciclo di sotto-obiettivi, ad esempio:
\begin{enumerate}
    \item Cercare la spazzatura;
    \item Andare verso la spazzatura trovata;
    \item Prendere la spazzatura appena raggiunta;
    \item Cercare il bidone
    \item Andare verso il bidone trovato
    \item Riciclare la spazzatura
\end{enumerate}

Utilizzando il formalismo di Jason, i sotto-obiettivi elencati sono associabili a dei \textit{"plans"} di un agente. Al di fuori del primo plan (Cercare la spazzatura), normalmente attivato dal \textit{"goal"} principale presente nell'agente, i successivi possono essere attivati da una percezione ricevuta dall'agente. Ad esempio, in risposta alla ricerca del bidone è possibile notificare la scoperta di un bidone nelle vicinanze e, di conseguenza, fare in modo che l'agente utilizzi il plan "Andare verso il bidone trovato".

\begin{figure}[H]
\centering
\includegraphics[width=0.7\textwidth]{figures/Esempio_relazione.png}
\caption{Esempio di comunicazione tra mente e corpo}
\end{figure}

L'immagine mostra il flusso ordinato di interazioni per l'esempio appena descritto. La mente per svolgere il plan \textit{"Cercare il bidone"} vuole inviare al proprio corpo l'azione \textit{"Cerca bidone"}. La richiesta di svolgere l'azione inizia dall'utilizzo dell'operazione presente nell'artefatto personale dell'agente\footnote{previa associazione dei due}, in possesso del canale per comunicare con il middleware.

\begin{figure}[H]
\centering
\includegraphics[width=\textwidth]{figures/da_Agente_a_Game_Object.png}
\caption{Fase in invio azione a GameObject}
\end{figure}

L'artefatto quindi invia il messaggio al middleware che si occupa di inoltrare le informazioni al Game Object. Alla ricezione delle stesse, l'entità corpo (GameObject) attua l'azione richiesta e risponde alla mente (Agente) inviandogli la percezione generata, ad esempio \textit{"trovato(nomeBidone,posizione)"}.

\begin{figure}[H]
\centering
\includegraphics[width=\textwidth]{figures/da_Game_Object_a_Agente.png}
\caption{Fase di invio percezione ad Agente}
\end{figure}

A questo punto la percezione viene mandata al middleware che, a sua volta, la inoltrerà all'artefatto collegato. L'artefatto, nel momento in cui riceve la percezione, aggiunge quest'ultima alle sue proprietà osservabili che, automaticamente, aggiorneranno la BeliefBase dell'agente. L'ultimo passaggio rappresenta il punto cruciale per completare il collegamento tra corpo e mente dato che in questa maniera l'agente ha ricevuto la percezione dal proprio corpo. 

\medskip

Si intende inoltre lasciare aperta la possibilità, da parte del corpo, di inviare percezioni non come reazione ad azioni eseguite dalla mente, dato che il collegamento WebSocket, una volta effettuato, rimane attivo per tutta la durata di vita dell'entità. Ad esempio, in caso di contatto con un oggetto nella scena Unity, il corpo deve essere in grado di mandare una percezione del tipo \textit{"toccata(nome\_oggetto,posizione)"} alla propria mente senza bisogno di stimoli.



\chapter{Conclusioni}

Lo studio effettuato ha identificato le linee guida per collegare le astrazioni di Game Engine (GE) con quelle del Sistema Multi-Agente (MAS) con l'obiettivo di utilizzare la scena della GE come layer di modellazione dell'ambiente per il MAS. La definizione di un middleware (Synapsis) di collegamento ha permesso di non effettuare modifiche sostanziali sulle astrazioni dei due sistemi.

\medskip

I successivi passi comprenderanno la ricerca della tecnologia per realizzare il middleware, della tecnologia per collegare le tre componenti (GE, MAS, Synapsis) del sitema delineato e realizzare le librerie di supporto per GE e MAS, tenendo conto dei loro diversi modelli computazionali. 


%\chapter*{Materiale online} \label{materiale_online}
\markboth{MATERIALE ONLINE}{}
\section*{Synapsis}
Repository con middleware e librerie disponibili all'indirizzo:\\
\href{https://gitlab.com/lucapascu/mas-ge-middleware}{https://gitlab.com/lucapascu/mas-ge-middleware}

\medskip

\section*{Progetto JaCaMo}
Repository con caso di studio e ambiente di test disponibili all'indirizzo:\\
\href{https://gitlab.com/lucapascu/mas-ge-jacamo}{https://gitlab.com/lucapascu/mas-ge-jacamo}

\medskip

\section*{Progetto Unity}
Repository con caso di studio e ambiente di test disponibili all'indirizzo:\\
\href{https://gitlab.com/lucapascu/mas-ge-jacamo}{https://gitlab.com/lucapascu/mas-ge-jacamo}


%----------------------------------------------------------------------------------------
%	THESIS CONTENT - APPENDICES
%----------------------------------------------------------------------------------------

\appendix % Cue to tell LaTeX that the following "chapters" are Appendices

% Include the appendices of the thesis as separate files from the Appendices folder
% Uncomment the lines as you write the Appendices

%\include{appendici/appendice}
%\include{Appendices/AppendixB}
%\include{Appendices/AppendixC}

% \appendix

%%%%%%%%%%%%%%%%%%%%%%%%%
% inizio parte finale del documento
%
% eventuali appendici, bibliografia obbligatoria,
% eventuale lista delle tabelle e delle figure (nel caso decommentare la riga con i comandi \listoffigures e \listoftables)
%%%%%%%%%%%%%%%%%%%%%%%%%
\backmatter

%\input{./Appendice/appendice.tex}
%\input{./Bibliografia/bibliografia.tex}

%\listoftodos[Lista Note]
\listoffigures
%\listoftables
%\lstlistoflistings
\bibliography{biblio}
\bibliographystyle{abbrv}

% chiusura del documento
\end{document}
