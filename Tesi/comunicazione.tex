\section{Comunicazione}

\'E necessario precisare che nel MAS, l'interpretazione dell'ambiente da parte di un agente si basa su concetti semantici che formano un'astrazione del mondo fisico in virtuale.
Le rappresentazioni di questi stessi concetti nella GE spesso hanno un livello di astrazione diverso rispetto a ciò che è più adatto per gli agenti.
Ad esempio, il concetto di "persona seduta su una sedia", associabile ad un belief del tipo \textit{"seduta(persona,sedia)"} di un agente,
può essere rappresentato nella GE dalla posizione di un personaggio in prossimità di una sedia in combinazione con le posizioni di ciascun componente, formando una posizione seduta.

\medskip

\subsection{Protocollo messaggi} \label{protocollo_messaggi}

Il protocollo di comunicazione prevede un solo messaggio per ogni interazione.
Bisogna tenere presente che, se un messaggio viene inviato dal lato MAS sicuramente sarà un azione; se un messaggio viene inviato dal lato GE, sarà
invece una percezione.

\medskip

Il messaggio è così strutturato:
\begin{itemize}
   \item \textbf{Sender}: indica il nome dell'entità (corpo o mente) che invia il messaggio;
   \item \textbf{Receiver}: indica il nome dell'entità (corpo o mente) che deve ricevere il messaggio;
   \item \textbf{Content}: contenuto principale del messaggio, corrisponde all'identificativo dell'azione o percezione inviata, ad esempio \textit{touched} e \textit{search};
   \item \textbf{Params}: array di parametri associati alla percezione o azione. Data la diversa tipologia di linguaggi presenti nel sistema è stato deciso di supportare solo i tipi primitivi quali stringhe, numeri e booleani;
   \item \textbf{TimeStats}: array ordinato di statistiche temporali.
\end{itemize}

La struttura del messaggio ha contribuito ad una facile associazione al formato JSON \cite{json} generato utilizzando le librerie GSON \cite{gson}, all'interno di Synapsis e JaCaMo, e Json.Net \cite{json_net}, all'interno di Unity, che permettono di serializzare e deserializzare una specifica classe a Json e viceversa.

\lstinputlisting[label={messaggioJSON},caption={JSON di un generico messaggio},language=json]{code/Message.json}

Il listato \ref{messaggioJSON} rappresenta un esempio di messaggio, in formato JSON, che mente e corpo si possono scambiare all'interno del sistema.

\subsubsection{Interpretare i dati statistici}

La separazione tra le diverse parti del sistema (Game Engine, Middleware e Sistema Multi-Agente) ha portato alla necessità di inserire delle statistiche temporali per venire a conoscenze dei tempi di trasferimento dei messaggi.

\medskip

Per interpretare i dati statistici è necessario conoscere la loro natura: è stato utilizzato il timestamp (in millisecondi) dal 1 Gennaio 1970 UTC\footnote{link al metodo \href{https://docs.oracle.com/en/java/javase/12/docs/api/java.base/java/lang/System.html}{Java}. Per C\# è stato realizzato un metodo ad hoc che funziona allo stesso modo.} per segnare il momento esatto di invio/ricezione del messaggio.

\medskip

Dato, per esempio, il precedente array di tempistiche, è possibile calcolare i tempi in questa modalità:

\begin{enumerate}
    \item \textbf{Trasmissione da MAS/GE a Synapsis(attore)}: differenza tra il secondo ed il primo valore, quindi: \textit{1563478689310 - 1563478689301 = 9 mills};
    \item \textbf{Trasmissione tra attori interni a Synapsis}: differenza tra il terzo ed il secondo valore, quindi: \textit{1563478689311 - 1563478689310 = 1 mills};
    \item \textbf{Trasmissione da Synapis a MAS/GE}: differenza tra il quarto ed il terzo valore, quindi: \textit{1563478689320 - 1563478689311 = 9 mills}.
\end{enumerate}

Il tempo totale di trasmissione del messaggio è calcolabile sommando i risultati precedentemente ottenuti, in questo caso 19 mills.