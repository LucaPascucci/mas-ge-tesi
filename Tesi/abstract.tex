Questa tesi studia lo stato di integrazione tra Game Engine (GE) e Sistemi Multi-Agente (MAS), per proporre un'infrastruttura generica utilizzabile per diversi scenari di associazione e comunicazione tra MAS e GE rispettandone il disaccoppiamento e l'integrità concettuale delle loro astrazioni. 
L’astrazione primaria è l’entità, intesa come oggetto divisibile in due parti, mente e corpo, che collegate riescono a trasmettersi informazioni, utilizzate dalla mente per raggiungere i propri obiettivi e dal corpo per diventare "attivo" nell’ambiente in cui si trova.
Il compito principale della GE (Unity) è di definire il corpo dell’entità rendendola fisicamente cosciente dell’ambiente che la circonda attraverso l’invio di percezioni alla mente, nello specifico di utilizzano Script e GameObject, componenti principali di Unity. Le astrazioni presenti nel MAS (JaCaMo) vengono utilizzate per realizzare la mente dell’entità, nello specifico viene fatto uso di agenti, componente autonoma, e artefatti per creare un primo canale di comunicazione.
Il middleware, componente realizzato per associare le parti dell’entità e metterle in comunicazione, è stato sviluppato utilizzando il framework Play e contiene un metodo di prototipazione rapida per non obbligare lo sviluppatore a realizzare in simultanea lo scenario in entrami i sistemi.
L’obiettivo finale è quindi quello di mettere a disposizione dello sviluppatore un’infrastruttura preliminare con la quale è possibile realizzare scenari relativamente complessi.