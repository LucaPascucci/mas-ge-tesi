\chapter{Sommario}

L'ambiente, o scena secondo la terminologia dei videogiochi, è la parte fondamentale di numerosi giochi dato che permette al giocatore di entrare in sinergia con il tipo, lo scopo e le modalità del gioco. Un esempio lampante è un FPS\footnote{First-person shooter = sparatutto in prima persona} dove la scena è solitamente vista in prima persona dal videogiocatore e la presenza di elementi con i quali interagire crea interesse nel giocatore ad esplorare l'ambiente attorno a lui.

\medskip

In questa relazione si intende studiare lo stato di integrazione tra Game Engine (GE) e Sistemi Multi-Agente (MAS) con l'obiettivo di utilizzare la scena, realizzabile con una GE, come layer di modellazione dell'ambiente per il MAS. La prima fase approfondisce le astrazioni e funzionalità messe a disposizione nella GE, in particolare Unity, per realizzare, modellare ed interagire con l'ambiente. La seconda fase studia lo stato dell'arte delle integrazioni e identifica delle linee guida di collegamento delle astrazioni dei due sistemi.
Per concludere viene definita la struttura del sistema per effettuare collegamento dei due sistemi, mantendoli separati e quindi senza modificare le loro astrazioni e funzionalità.