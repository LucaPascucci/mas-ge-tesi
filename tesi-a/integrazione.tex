\subsection{Integrazione}

Sono già presenti esempi di integrazione tra GE e MAS che concentrano la propria attenzione su obiettivi specifici a livello tecnologico, piuttosto che sulla creazione di un'infrastruttura orientata agli agenti basata sul gioco per scopi generici. Per esempio:

\begin{itemize}
    \item QuizMASter \cite{5763564} concentrato sull'astrazione degli agenti collegando gli agenti MAS ai personaggi dei motori di gioco, nel contesto dell'apprendimento educativo
    \item CIGA \cite{ciga} considera sia la modellazione degli agenti che quella dell'ambiente, per agenti virtuali generici in ambienti virtuali
    \item GameBots \cite{gamebots} concentrato sull'astrazione dell'agente, ma considera anche l'ambiente fornendo un framework di sviluppo e un runtime per i test di sistemi multi-agente in ambienti virtuali
    \item UTSAF \cite{utsaf} si concentra sulla modellistica ambientale nel contesto di simulazioni distribuite in ambito militare\footnote{Gli agenti vengono considerati, ma solo come mezzo di integrazione tra diverse piattaforme di simulazione, non nel contesto del GE sfruttato per il rendering di simulazione}
\end{itemize}

Sebbene rappresentino chiaramente esempi di integrazione (parzialmente) riuscita di MAS in GE, i lavori sopra elencati presentano alcune carenze rispetto all'obiettivo che perseguiamo in questo documento.

\smallskip

Solamente CIGA rappresenta un'eccezione che riconosce il divario concettuale tra MAS e GE, e propone soluzioni per affrontarlo (anche se a livello tecnologico). L'unico strato preso in considerazione nel perseguimento dell'integrazione è quello tecnologico - nessun modello, nessuna architettura, nessun linguaggio. All'interno di QuizMASter, UTSAF e GameBot (in una certa misura) l'integrazione è realizzata per specifico obiettivo, e la maggior parte degli approcci fornisce ai programmatori alcune astrazioni per trattare con agenti e ambiente, ma nessuna attenzione viene data alle astrazioni sociali \cite{gamemas-woa2016}.
