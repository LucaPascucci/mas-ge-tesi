\subsection{Sistema Multi-Agente} \label{sistema_multi-agente}

La crescente complessità nell'ingegnerizzazione dei sistemi software ha portato alla necessità di modelli e astrazioni in grado di rendere più facile la loro progettazione, lo sviluppo e il mantenimento. In questa direzione, la computazione orientata agli agenti viene in aiuto agli ingegneri ed informatici per costruire sistemi complessi, virtuali o artificiali permettendo una loro agevole e corretta gestione \cite{mas-as-complex-systems}.

\medskip

In particolare, la ricerca e le tecnologie per MAS hanno introdotto nuove astrazioni per affrontare la complessità durante la progettazione di sistemi o applicazioni composte da individui che non agiscono più da soli ma all'interno di una società. Le tecnologie e i modelli agent-oriented sono attualmente diventati una potente tecnica in grado di affrontare molti problemi che vengono alla luce durante la progettazione di sistemi computer-based in termini di entità che condividono caratteristiche quali l'autonomia, l'intelligenza, la distribuzione, l'interazione, la coordinazione, etc.

\medskip

L’ingegnerizzazione dei MAS si occupa infatti di costruire sistemi complessi dove più entità autonome chiamate agenti cercano di raggiungere in maniera proattiva i loro scopi sfruttando le interazioni tra di essi (come una società), e con l'ambiente circostante. Questo modello può essere visto come un paradigma general-purpose, il quale prevede l'utilizzo di tecnologie agent-oriented in diversi scenari applicativi \cite{aose-jaamas9}.

\medskip

Un MAS fornisce agli sviluppatori e ai designer tre astrazioni principali:
\begin{itemize}
    \item Agenti: Le entità autonome che compongono il sistema. Sono in grado di comunicare e possono essere intelligenti, dinamici, e situati;
    \item Società: Rappresenta un gruppo di entità il cui comportamento emerge dall'interazione tra i singoli elementi;
    \item Ambiente: Il "contenitore" in cui gli agenti sono immersi e con il quale questi ultimi possono interagire, modificandolo. La caratteristica degli agenti di essere situati nell'ambiente in cui si trovano permette loro di percepire e produrre cambiamenti su di esso.
\end{itemize}
