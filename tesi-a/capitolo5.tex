\chapter{Conclusioni e Sviluppi Futuri}

Il sistema sviluppato rappresenta un'implementazione preliminare di integrazione tra MAS e GE, e sfruttando le potenzialità offerte da Unity, si ha la conferma, ancora una volta, di come le due tecnologie – quella dei Game Engine e quella dei MAS – si sposino bene insieme. L'obiettivo primario di migliorare lo stato dell'arte dell'integrazione tra GE e MAS è stato raggiunto, mettendo a disposizione del programmatore un middleware e librerie per Unity e JaCaMo con le quali è possibile realizzare scenari relativamente complessi. Rimangono, tuttavia, margini di miglioramento e ulteriori studi da compiere.

\medskip

Per quanto riguarda il sistema nella sua interezza sarebbe interessante inserire SpatialTuples come mezzo abilitante la coordinazione, già utilizzato nei lavoro \cite{amslaurea16100} ed utilizzare il Web Socket Secure (WSS)\footnote{connessione criptata attraverso TLS/SSL} per migliorare la sicurezza complessiva delle comunicazioni. Sempre dal punto di vista della comunicazione sarebbe da aggiungere la possibilità di inviare strutture complesse (oggetti) nei parametri dei messaggi scambiati.

\medskip

Per quanto riguarda il middleware sarebbe sicuramente utile la realizzazione di una interfaccia grafica nella quale visualizzare le varie statistiche di invio e ricezione messaggi (raccolta delle tempistiche già presente), le entità presenti collegate, quali sono le entità mock, al fine di avere un vero e proprio centro di controllo del middleware.

\medskip 

In merito alla diffusione del sistema si potrebbero realizzare nuove librerie per aumentare l'estensione a più GE e/o MAS, cosi da non renderelo ad uso esclusivo di Unity e JaCamo ma, ad esempio, anche per altre GE (Unreal Engine, GameMaker, CryEngine) ed altri Multi-Agent System. A tal proposito si dovrebbbe aggiornare la libreria per JaCaMo supportandola alla sua ultima versione \href{https://sourceforge.net/projects/jacamo/files/version-0/}{JaCaMo 0.8}. 

\medskip 

Un altro interessante spunto di riflessione sui possibili lavori futuri riguarda la distribuzione: tutte le parti che compongono il sistema offrono diverse modalità di eseguire gli applicativi su diversi dispositivi. A tal proposito, Unity offre nativamente il supporto al Multiplayer che potrebbe essere sfruttato per indagare più a fondo in questa direzione.