\section{Terminologia}

La sinapsi (o giunzione sinaptica) (dal greco synàptein, vale a dire "connettere") è una struttura altamente specializzata che consente la comunicazione delle cellule del tessuto nervoso tra loro (neuroni) o con altre cellule (cellule muscolari, sensoriali). Nello specifico la sinapsi neuromuscolare rappresenta la giunzione tra neurone motore e muscolo a livello della placca motrice, ove ha luogo la trasmissione dell'impulso con le modalità delle sinapsi chimiche: lo spazio extracellulare della sinapsi neuromuscolare è detto chiave sinaptica \cite{treccani}.
La semplice associazione tra l'obiettivo di questo percorso e la parola sopra definita ha portato a denominare il middleware "Synapsis"\footnote{Traduzione in inglese del termine italiano sinapsi.}.

\subsection{Entità}

Successivamente nella trattazione verrà fatto uso del termine "entità" che generalmente viene intesa come insieme di elementi dotati di proprietà comuni dal punto di vista dell’applicazione considerata \cite{treccani}.
Concettualmente, in questo dominio, l'entità viene intesa come oggetto divisibile in due parti, mente e corpo, che collegate riescono a trasmettersi informazioni, utilizzate dalla mente per raggiungere i propri obiettivi e dal corpo per diventare "attivo" nell'ambiente in cui si trova.

\subsection{Mente}

La nozione di mente può essere caratterizzata da alcuni punti chiave fondamentali:
\begin{itemize}
   \item autonomia;
   \item interazione;
   \item obiettivi.
\end{itemize}
In altre parole, una mente può essere pensata come un componente software autonomo che interagisce con l'ambiente per svolgere i propri compiti.
I punti sopra elencati rendono facile l'associazione della mente al concetto di Agente, spiegato nella sezione \ref{sistema_multi-agente}, poiché questa entità del Sistema Multi-Agente (MAS) ingloba astrazioni simili a quelle illustrate nella sezione \ref{jason}.

\subsection{Corpo}

Corpo è un termine generico che indica qualsiasi porzione limitata di materia, cui si attribuiscono, in fisica, le proprietà di estensione, divisibilità, impenetrabilità \cite{treccani}.
In questa trattazione è associabile alla nozione di GameObject di Unity, spiegata nella sezione \ref{unity}, utilizzata per avere una rappresentazione fisica dell'entità da realizzare.

\subsection{Azione}

Nel suo significato più generale è intesa come attività od operazione posta in essere da un determinato soggetto \cite{treccani}.
In questo studio, si considera come "azione" un certo gesto richiesto dalla mente che può essere associato ad una operazione eseguita dal corpo, ad esempio, nel
caso di un'azione del tipo \textit{"vai a (posizione)"}, richiesta dalla mente, corrisponde il movimento del corpo nell'ambiente verso la posizione indicata.

\subsection{Percezione}

La percezione è un atto cognitivo mediato dai sensi con cui si avverte la realtà di un determinato oggetto e che implica un processo di organizzazione e interpretazione \cite{treccani}.

\medskip

In questo lavoro, la percezione si collega ad una certa sensazione rilevata dal corpo ed inviata alla mente per portarla a conoscenza di questa nuova informazione, ad esempio, nel caso del raggiungimento della posizione richiesta in precedenza, il corpo trasmette la percezione \textit{"arrivato (posizione)"} che informa la mente del completamento dell'operazione.

\medskip

Esiste inoltre, da parte del corpo, la possibilità di inviare percezioni "libere" ossia non associate a risposta di un'azione inviata dalla mente. Un semplice esempio è il contatto del corpo con una qualsiasi altra entità nell'ambiente che corrisponde all'invio di una percezione del tipo \textit{"toccato(nome\_entità)"}.

\subsection{Struttura di un'entità} \label{struttura_entita1}

\begin{figure}[H]
   \centering
   \includegraphics[width=8cm]{figures/Entita_struttura.png}
   \caption{Struttura di una generica entità}
   \label{entita_struttura}
\end{figure}

La figura \ref{entita_struttura} rappresenta la struttura di una generica entità, dove:

\begin{itemize}
   \item Il corpo esegue azioni e, in risposta a queste ultime, oppure, a seguito di determinati eventi esterni, trasmette le proprie percezioni alla mente.
   \item La mente elabora le percezioni per decidere quali azioni far svolgere al proprio corpo.
\end{itemize}


